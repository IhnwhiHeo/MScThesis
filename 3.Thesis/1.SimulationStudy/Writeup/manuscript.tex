% Options for packages loaded elsewhere
\PassOptionsToPackage{unicode}{hyperref}
\PassOptionsToPackage{hyphens}{url}
%
\documentclass[
]{article}
\usepackage{lmodern}
\usepackage{amssymb,amsmath}
\usepackage{ifxetex,ifluatex}
\ifnum 0\ifxetex 1\fi\ifluatex 1\fi=0 % if pdftex
  \usepackage[T1]{fontenc}
  \usepackage[utf8]{inputenc}
  \usepackage{textcomp} % provide euro and other symbols
\else % if luatex or xetex
  \usepackage{unicode-math}
  \defaultfontfeatures{Scale=MatchLowercase}
  \defaultfontfeatures[\rmfamily]{Ligatures=TeX,Scale=1}
\fi
% Use upquote if available, for straight quotes in verbatim environments
\IfFileExists{upquote.sty}{\usepackage{upquote}}{}
\IfFileExists{microtype.sty}{% use microtype if available
  \usepackage[]{microtype}
  \UseMicrotypeSet[protrusion]{basicmath} % disable protrusion for tt fonts
}{}
\makeatletter
\@ifundefined{KOMAClassName}{% if non-KOMA class
  \IfFileExists{parskip.sty}{%
    \usepackage{parskip}
  }{% else
    \setlength{\parindent}{0pt}
    \setlength{\parskip}{6pt plus 2pt minus 1pt}}
}{% if KOMA class
  \KOMAoptions{parskip=half}}
\makeatother
\usepackage{xcolor}
\IfFileExists{xurl.sty}{\usepackage{xurl}}{} % add URL line breaks if available
\IfFileExists{bookmark.sty}{\usepackage{bookmark}}{\usepackage{hyperref}}
\hypersetup{
  pdftitle={Missing the Point: Convergence of Multiple Imputation using Chained Equations Algorithms},
  pdfauthor={Hanne Oberman},
  hidelinks,
  pdfcreator={LaTeX via pandoc}}
\urlstyle{same} % disable monospaced font for URLs
\usepackage[margin=1in]{geometry}
\usepackage{graphicx,grffile}
\makeatletter
\def\maxwidth{\ifdim\Gin@nat@width>\linewidth\linewidth\else\Gin@nat@width\fi}
\def\maxheight{\ifdim\Gin@nat@height>\textheight\textheight\else\Gin@nat@height\fi}
\makeatother
% Scale images if necessary, so that they will not overflow the page
% margins by default, and it is still possible to overwrite the defaults
% using explicit options in \includegraphics[width, height, ...]{}
\setkeys{Gin}{width=\maxwidth,height=\maxheight,keepaspectratio}
% Set default figure placement to htbp
\makeatletter
\def\fps@figure{htbp}
\makeatother
\setlength{\emergencystretch}{3em} % prevent overfull lines
\providecommand{\tightlist}{%
  \setlength{\itemsep}{0pt}\setlength{\parskip}{0pt}}
\setcounter{secnumdepth}{-\maxdimen} % remove section numbering

\title{Missing the Point: Convergence of Multiple Imputation using Chained
Equations Algorithms}
\author{Hanne Oberman}
\date{16-3-2020}

\begin{document}
\maketitle

\hypertarget{introduction}{%
\section{Introduction}\label{introduction}}

Feedback:

\begin{itemize}
\item
  Focus on convergence, not evaluation of MI in general.
\item
  Combine Theoretical Background and Intro.
\end{itemize}

At some point, any scientist conducting statistical analyses will run
into a missing data problem (Allison 2001). Missingness is problematic
because statistical inference cannot be performed on incomplete data
without employing \emph{ad hoc} solutions (e.g., list-wise deletion),
which may yield wildly invalid results (Van Buuren 2018). A popular
answer to the ubiquitous problem of missing information is to use the
framework of multiple imputation (MI), proposed by Rubin (1987). MI is
an iterative algorithmic procedure in which missing datapoints are
`imputed' (i.e.~filled in) several times. The variability between
imputations is used to reflect how much uncertainty in the inference is
introduced by the missingness. Therefore, MI can provide valid
inferences despite missing information (\textbf{maybe add that this
refers to CIs/p-values}).

To obtain valid inferences with MI, the variability between imputations
should be properly represented (Rubin 1987; Van Buuren 2018). If this
variability is under-estimated, confidence intervals around estimates
will be too narrow, which can yield spurious results. Over-estimation of
the variance between imputations results in unnecessarily wide
confidence intervals, which can be costly because it lowers the
statistical power (\textbf{maybe add why this is costly}). Since both of
these situations are undesirable, imputations and their variability
should be evaluated. Evaluation measures, however, are currently missing
or under-developed in MI software, like the world-leading \texttt{mice}
package (Van Buuren and Groothuis-Oudshoorn 2011) in \texttt{R} (R Core
Team 2019). The goal of this research project is to develop novel
methodology and guidelines for evaluating MI methods. These tools will
subsequently be implemented in an interactive evaluation framework for
multiple imputation, which will aid applied researchers in drawing valid
inference from incomplete datasets.

This note provides the theoretical foundation towards the diagnostic
evaluation of multiple imputation algorithms. For reasons of brevity, we
only focus on the MI algorithm implemented in \texttt{mice} (Van Buuren
and Groothuis-Oudshoorn 2011). The convergence properties of this MI
algorithm are investigated through model-based simulation.\footnote{All
  programming code used in this note is available from
  {[}github.com/gerkovink/shinyMice/simulation{]}\{\url{https://github.com/gerkovink/ShinyMICE/simulation}\}.}
The results of this simulation study are guidelines for assessing
convergence of MI algorithms.

\hypertarget{terminology}{%
\subsubsection{Terminology}\label{terminology}}

This note follows notation and conventions of \texttt{mice} (Van Buuren
and Groothuis-Oudshoorn 2011). Basic familiarity with MI methodology is
assumed. For the theoretical foundation of MI, see Rubin (1987). For an
accessible and comprehensive introduction to MI from an applied
perspective, see e.g.~Van Buuren (2018).

Let \(Y\) denote an \(n \times p\) matrix containing the data values on
\(p\) variables for all \(n\) units in a sample. The collection of
observed data values in \(Y\) is denoted by \(Y_{obs}\), and will be
referred to as `incomplete' or `observed' data. The missing part of
\(Y\) is denoted by \(Y_{mis}\). Which parts of \(Y\) are missing is
determined by the `missingness mechanism'.\\
This note only considers a `missing completely at random' (MCAR)
mechanism, where the probability of being missing is equal for all
\(n \times p\) cells in \(Y\) (Rubin 1987).

Figure 1 provides an overview of the steps involved with MI---from
incomplete data, to \(m\) completed datasets, to \(m\) estimated
quantities of interest (\(\hat{Q}\)s), to a single pooled estimate
\(\bar{Q}\). This note focuses on the algorithmic properties of the
imputation step.

The MI algorithm in \texttt{mice} has an iterative nature. For each
missing datapoint in \(Y_{mis}\), \(m\) `chains' of potential values are
sampled. Only the ultimate sample that each chain lands on is imputed.
The number of iterations per chain will be denoted with \(T\), where
\(t\) varies over the integers \(1, 2, \dots, T\). This is repeated
\(m\) times. Imputed values are the ultimate samples of a `chain' of For
each imputed value (\(t = T\)), a s for each missing datapoint in
\(Y_{mis}\), . Each of the \(m\) chains starts with an initial value,
drawn randomly from \(Y_{obs}\). The chains are terminated after a
predefined number of iterations. , and subsequently used in the analysis
and pooling steps. The collection of samples between the initial value
(at \(t=1\)) and the imputed value (at \(t=T\)) will be referred to as
an `imputation chain'.

\hypertarget{methods}{%
\section{Methods}\label{methods}}

\hypertarget{results}{%
\section{Results}\label{results}}

Feedback:

\begin{itemize}
\item
  Remove the table.
\item
  Add more info about figure legends and axes.
\end{itemize}

\hypertarget{discussion}{%
\section{Discussion}\label{discussion}}

\hypertarget{references}{%
\section*{References}\label{references}}
\addcontentsline{toc}{section}{References}

\hypertarget{refs}{}
\leavevmode\hypertarget{ref-alli02}{}%
Allison, Paul D. 2001. \emph{Missing Data}. Sage publications.

\leavevmode\hypertarget{ref-R}{}%
R Core Team. 2019. \emph{R: A Language and Environment for Statistical
Computing}. Vienna, Austria. \url{https://www.R-project.org/}.

\leavevmode\hypertarget{ref-rubin87}{}%
Rubin, Donald B. 1987. \emph{Multiple Imputation for Nonresponse in
Surveys}. Wiley Series in Probability and Mathematical Statistics
Applied Probability and Statistics. New York, NY: Wiley.

\leavevmode\hypertarget{ref-buur18}{}%
Van Buuren, Stef. 2018. \emph{Flexible Imputation of Missing Data}.
Chapman; Hall/CRC.

\leavevmode\hypertarget{ref-mice}{}%
Van Buuren, Stef, and Karin Groothuis-Oudshoorn. 2011. ``Mice:
Multivariate Imputation by Chained Equations in R.'' \emph{Journal of
Statistical Software} 45 (1): 1--67.
\url{https://doi.org/10.18637/jss.v045.i03}.

\end{document}
